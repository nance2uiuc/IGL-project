\documentclass{article}

\usepackage[margin=1in]{geometry}
\usepackage{graphicx}

\begin{document}
	
	\title{{\Large IGL Brochure - Exploring Entropy and Complexity of Curves on a Surface}}
	\author{Spencer Dowdall (Faculty mentor),
		Joe Nance,
		Victor Yang,
		Xiaolong Han,
		Yohan Kang}
	\date{November 14, 2014}
	\maketitle
	
	Each closed curve $\gamma$ with a base point $pt$ on a surface $S$ determines a natural homeomorphism $\mathcal{P}(\gamma)$ of the surface $S\setminus\{pt\}$ via \textbf{point-pushing}:
	
	\begin{figure}[h]
		\begin{center}
			\includegraphics[width=12cm]{Ex_Point-Pushing_Homeomorphism.pdf}
			\caption{{\bf Point-push} around $\gamma$ by dragging basepoint $pt$ once around $\gamma$.}
		\end{center}
	\end{figure}
	
	The goal of our project is to study point-pushing homeomorphisms on the genus-$g$ surface, and to relate their topological entropy to their combinatorial complexity. To this end, we developed a computer program that 
	randomly generates closed curves $\gamma$ on the surface, and then calculates the self-intersection number $\iota(\gamma)$ (measuring the combinatorial complexity) of the curve and the dilatation number (measuring the topological entropy) of the resulting point-pushing homeomorphism $\mathcal{P}(\gamma)$. We then statistically analyzed the results by calculating the distributions of the dilatation and self-intersection number data and looking for correlations between these numbers.
	
\end{document}